\documentclass[
11pt, % The default document font size, options: 10pt, 11pt, 12pt
%oneside, % Two side (alternating margins) for binding by default, uncomment to switch to one side
english, % ngerman for German
singlespacing, % Single line spacing, alternatives: onehalfspacing or doublespacing
%draft, % Uncomment to enable draft mode (no pictures, no links, overfull hboxes indicated)
%nolistspacing, % If the document is onehalfspacing or doublespacing, uncomment this to set spacing in lists to single
liststotoc, % Comment to remove the list of figures/tables/etc from the table of contents
toctotoc, % Comment to remove the main table of contents from the table of contents
%parskip, % Uncomment to add empty space between paragraphs instead of identation. Keep in mind this is wasteful when printed
headsepline, % Comment to remove the line under the header
%chapterinoneline, % Uncomment to place the chapter title next to the number on one line
consistentlayout, % Comment to change the layout of the declaration, abstract and acknowledgements pages to be different than the default layout
bachelor, % Comment-out if you are writing a Master Thesis
]{style} % The class file specifying the document structure

\usepackage[utf8]{inputenc} % Required for inputting international characters
\usepackage[T1]{fontenc} % Output font encoding for international characters
\usepackage{mathpazo} % Use the Palatino font by default

\usepackage[backend=bibtex,style=authoryear,natbib=true]{biblatex} % Use the bibtex backend with the authoryear citation style (which resembles APA)

\addbibresource{lit.bib} % The filename of the bibliography

\usepackage[autostyle=true]{csquotes} % Required to generate language-dependent quotes in the bibliography

\usepackage{hyperref}
\usepackage{varioref}
\usepackage{cleveref} % For automatically referencing images, tables, listings, relative pages, etc...

\usepackage[inline]{enumitem} % For inlined enumerations and itemizations
\setenumerate[1]{label={(\arabic*)}} % Change enumeration labels from 1. to (1)

% margins
\geometry{
	paper=a4paper, % Change to letterpaper for US letter
	inner=2.5cm, % Inner margin
	outer=3.8cm, % Outer margin
	bindingoffset=.5cm, % Binding offset
	top=1.5cm, % Top margin
	bottom=1.5cm, % Bottom margin
	%showframe, % Uncomment to show how the type block is set on the page
}

\thesistitle{Thesis Title} % Your thesis title, this is used in the title, print it elsewhere with \ttitle
\firstexaminer{Prof. Dr. Stefan Brunthaler} % Your first examiner's name, this is used in the title page, print it elsewhere with \fxname
\secondexaminer{Prof. Dr. Max Mustermann} % Your second examiner's name, this is used in the title page, print it elsewhere with \sxname
\supervisor{M.Sc. Max Mustermann} % Your supervisor's name, this is used in the title page, print it elsewhere with \supname
\author{John Smith} % Your name, this is used in the title page, print it elsewhere with \authorname
\matriculationnumber{1234567} % Your Matrikelnummer, this is used in the title page, print it elsewhere with \matnumber

\handindate{01.01.1900} % Your Abgabedatum (used in the title page and on the info sheet, print it elsewhere with \hdate)

\university{Universit\"at der Bundeswehr M\"unchen} % Your university's name, this is used in the title page, print it elsewhere with \univname
\department{Fakult\"at f\"ur Informatik} % Your department's name, this is used in the title page, print it elsewhere with \deptname

\AtBeginDocument{
\hypersetup{pdftitle=\ttitle} % Set the PDF's title to your title
\hypersetup{pdfauthor=\authorname} % Set the PDF's author to your name
\hypersetup{allcolors=blue} % Color all links/references in blue
\hypersetup{colorlinks=false} % Do not color the links in the printed PDF
}
