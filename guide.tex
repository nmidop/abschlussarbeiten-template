\chapter{\LaTeX{} Guide and Examples}\label{chap:guide}

% Define some commands to keep the formatting separated from the content 
\newcommand{\keyword}[1]{\textbf{#1}}
\newcommand{\tabhead}[1]{\textbf{#1}}
\newcommand{\code}[1]{\texttt{#1}}
\newcommand{\file}[1]{\texttt{\bfseries#1}}
\newcommand{\option}[1]{\texttt{\itshape#1}}

\section{Welcome}
Welcome to this \LaTeX{} Thesis Template, a nicely looking and easy to use template for writing a thesis using the \LaTeX{} typesetting system.

If you are writing a thesis (or will be in the future) and its subject is technical or mathematical (though it doesn't have to be), then creating it in \LaTeX{} is highly recommended as a way to make sure you can just get down to the essential writing without having to worry over formatting or wasting time arguing with your word processor.

\LaTeX{} is easily able to professionally typeset documents that run to hundreds or thousands of pages long. With simple mark-up commands, it automatically sets out the table of contents, margins, page headers and footers and keeps the formatting consistent. One of its main strengths is the way it can easily typeset mathematics, even \emph{heavy} mathematics. Even if those equations are the most horribly twisted and most difficult mathematical problems that can only be solved on a super-computer, you can at least count on \LaTeX{} to make them look good.

\section{Learning \LaTeX{}}

\LaTeX{} is not a \textsc{wysiwyg} (What You See is What You Get) program, unlike word processors such as Microsoft Word or Apple's Pages. Instead, a document written for \LaTeX{} is actually a simple, plain text file that contains \emph{no formatting}. You tell \LaTeX{} how you want the formatting in the finished document by writing in simple commands amongst the text, for example, if I want to use \emph{italic text for emphasis}, I write the \verb|\emph{text}| command and put the text I want in italics in between the curly braces. This means that \LaTeX{} is a \enquote{mark-up} language, very much like HTML.

\subsection{A (not so short) Introduction to \LaTeX{}}

If you are new to \LaTeX{}, there is a very good eBook -- freely available online as a PDF file -- called, \enquote{The Not So Short Introduction to \LaTeX{}}. The book's title is typically shortened to just \emph{lshort}. You can download the latest version (as it is occasionally updated) from here:
\url{http://www.ctan.org/tex-archive/info/lshort/english/lshort.pdf}

It is also available in several other languages. Find yours from the list on this page: \url{http://www.ctan.org/tex-archive/info/lshort/}

Making the effort now means you're not stuck learning the system when what you \emph{really} need to be doing is writing your thesis.

\subsection{A Short Math Guide for \LaTeX{}}

If you are writing a technical or mathematical thesis, then you may want to read the document by the AMS (American Mathematical Society) called, \enquote{A Short Math Guide for \LaTeX{}}. It can be found online here:
\url{http://www.ams.org/tex/amslatex.html}
under the \enquote{Additional Documentation} section towards the bottom of the page.

\subsection{Common \LaTeX{} Math Symbols}
There are a multitude of mathematical symbols available for \LaTeX{} and it would take a great effort to learn the commands for them all. The most common ones you are likely to use are shown on this page:
\url{http://www.sunilpatel.co.uk/latex-type/latex-math-symbols/}

You can use this page as a reference or crib sheet, the symbols are rendered as large, high quality images so you can quickly find the \LaTeX{} command for the symbol you need.

\subsection{\LaTeX{} on a Mac}
 
The \LaTeX{} distribution is available for many systems including Windows, Linux and Mac OS X. The package for OS X is called MacTeX and it contains all the applications you need -- bundled together and pre-customized -- for a fully working \LaTeX{} environment and work flow.
 
MacTeX includes a custom dedicated \LaTeX{} editor called TeXShop for writing your `\file{.tex}' files and BibDesk: a program to manage your references and create your bibliography section just as easily as managing songs and creating playlists in iTunes.

\section{Getting Started with this Template}

If you are familiar with \LaTeX{}, then you should explore the directory structure of the template and then proceed to place your own information into the \file{main.tex} and \file{settings.tex} files. 

If you are new to \LaTeX{} it is recommended that you carry on reading through the rest of the information in this document, as it contains useful examples that you can borrow while writing your thesis, and thus save valuable time.

\subsection{About this Template}

This \LaTeX{} Thesis Template is originally based and created around a \LaTeX{} style file created by Steve R.\ Gunn from the University of Southampton (UK), department of Electronics and Computer Science. You can find his original thesis style file at his site, here:
\url{http://www.ecs.soton.ac.uk/~srg/softwaretools/document/templates/}

Steve's \file{ecsthesis.cls} was then taken by Sunil Patel who modified it by creating a skeleton framework and folder structure to place the thesis files in. The resulting template can be found on Sunil's site here:
\url{http://www.sunilpatel.co.uk/thesis-template}

Sunil's template was made available through \url{http://www.LaTeXTemplates.com} where it was modified many times based on user requests and questions. Version 2.0 and onwards of this template represents a major modification to Sunil's template and is, in fact, hardly recognisable. The work to make version 2.0 possible was carried out by \href{mailto:vel@latextemplates.com}{Vel} and Johannes Böttcher.

Subsequently, Vasil Sarafov from Munich's Computer Systems Research Laboratory (\url{https://ucsrl.de}), Department of Informatics, University of the Bundeswehr Munich, forked the template from \url{http://www.LaTeXTemplates.com} and heavily modified it to satisfy the university's requirements.
Some parts of version 2.0 can still be found in the \file{style.cls} code.
However, large portions of the code has been overhauled.

\section{What this Template Includes}

\subsection{Folders}

This template comes as a single zip file that expands out to several files and folders. The folder names are mostly self-explanatory:

\keyword{root} -- the root directory contains almost everything: settings, styles, chapters, appendices.

\keyword{figures} -- this folder contains all figures for the thesis.

\keyword{build} -- this folder is created automatically when you build the \LaTeX{} source code using the \file{Makefile}.
The build process will put there all intermediate files, and finally will obtain \file{main.pdf} -- the compiled pdf document.

\subsection{Files}

Included are also several files, most of them are plain text and you can see their contents in a text editor:

\keyword{lit.bib} -- this is an important file that contains all the bibliographic information and references that you will be citing in the thesis for use with BibTeX. You can write it manually, but there are reference manager programs available that will create and manage it for you. Bibliographies in \LaTeX{} are a large subject and you may need to read about BibTeX before starting with this. Many modern reference managers will allow you to export your references in BibTeX format which greatly eases the amount of work you have to do.

\keyword{style.cls} -- this is an important file. It is the class file that tells \LaTeX{} how to format the thesis. You generally will not need to modify it (i.e. you can safely ignore it).

\keyword{main.pdf} -- this is your beautifully typeset thesis (in the PDF file format) created by \LaTeX{}. It is supplied in the PDF with the template and after you compile the template you should get an identical version.

\keyword{main.tex} -- this is the core source file that defines your thesis.
It includes everything else and tells \LaTeX{} how to produce your thesis as a PDF file.
It is heavily commented so you can read exactly what each line of code does and why it is there.

\keyword{settings.tex} -- self-explanatory. This file is included by \file{main.tex}.

\section{Filling in Your Information in the \file{main.tex} and \file{settings.tex} Files}\label{FillingFile}

You will need to personalise the thesis template and make it your own by filling in your own information. This is done by editing the \file{main.tex} and \file{settings.tex} files in a text editor or your favourite \LaTeX{} environment.

Fill out the information about yourself, and your thesis (e.g. its title).

When you have done this, save the file and recompile with \code{make}. 

\section{Thesis Features and Conventions}\label{ThesisConventions}

To get the best out of this template, there are a few conventions that you may want to follow.

One of the most important (and most difficult) things to keep track of in such a long document as a thesis is consistency. Using certain conventions and ways of doing things (such as using a Todo list) makes the job easier. Of course, all of these are optional and you can adopt your own method.

\subsection{Printing Format}

This thesis template is designed for double sided printing (i.e. content on the front and back of pages) as most theses are printed and bound this way. Switching to one sided printing is as simple as uncommenting the \option{oneside} option of the \code{documentclass} command at the top of the \file{settings.tex} file.

The text is set to 11 point by default with single line spacing, again, you can tune the text size and spacing should you want or need to using the options at the very start of \file{settings.tex}. The spacing can be changed similarly by replacing the \option{singlespacing} with \option{onehalfspacing} or \option{doublespacing}.

The paper size used in the template is A4, which is the standard size in Germany (and whole of Europe).

\subsection{References}

The \code{biblatex} package is used to format the bibliography and inserts references such as this one \parencite{Reference1}. The options used in the \file{settings.tex} file mean that the in-text citations of references are formatted with the author(s) listed with the date of the publication. Multiple references are separated by semicolons (e.g. \parencite{Reference2, Reference1}) and references with more than three authors only show the first author with \emph{et al.} indicating there are more authors (e.g. \parencite{Reference3}). This is done automatically for you. To see how you use references, have a look at the \file{guide.tex} source file, which is rendered below. Many reference managers allow you to simply drag the reference into the document as you type.

Scientific references should come \emph{before} the punctuation mark if there is one (such as a comma or period). The same goes for footnotes\footnote{Such as this footnote, here down at the bottom of the page.}. You can change this but the most important thing is to keep the convention consistent throughout the thesis. Footnotes themselves should be full, descriptive sentences (beginning with a capital letter and ending with a full stop). The APA6 states: \enquote{Footnote numbers should be superscripted, [...], following any punctuation mark except a dash.} The Chicago manual of style states: \enquote{A note number should be placed at the end of a sentence or clause. The number follows any punctuation mark except the dash, which it precedes. It follows a closing parenthesis.}

The bibliography is typeset with references listed in alphabetical order by the first author's last name. This is similar to the APA referencing style. To see how \LaTeX{} typesets the bibliography, have a look at the very end of this document (or just click on the reference number links in in-text citations).

\subsubsection{A Note on bibtex}

The bibtex backend used in the template by default does not correctly handle unicode character encoding (i.e. "international" characters). You may see a warning about this in the compilation log and, if your references contain unicode characters, they may not show up correctly or at all. The solution to this is to use the biber backend instead of the outdated bibtex backend. This is done by finding this in \file{settings.tex}: \option{backend=bibtex} and changing it to \option{backend=biber}. You will then need to delete all auxiliary BibTeX files and navigate to the template directory in your terminal (command prompt). Once there, simply type \code{biber main} and biber will compile your bibliography. You can then compile \file{main.tex} as normal and your bibliography will be updated. An alternative is to set up your LaTeX editor to compile with biber instead of bibtex, see \href{http://tex.stackexchange.com/questions/154751/biblatex-with-biber-configuring-my-editor-to-avoid-undefined-citations/}{here} for how to do this for various editors.

\subsection{Tables}

Tables are an important way of displaying your results, below is an example table which was generated with this code:

{\small
\begin{verbatim}
\begin{table}
\centering
\begin{tabular}{l l l}
\toprule
\tabhead{Groups} & \tabhead{Treatment X} & \tabhead{Treatment Y} \\
\midrule
1 & 0.2 & 0.8\\
2 & 0.17 & 0.7\\
3 & 0.24 & 0.75\\
4 & 0.68 & 0.3\\
\bottomrule\\
\end{tabular}
\end{table}
\caption{The effects of treatments X and Y on the four groups studied.}
\label{tab:treatments}
\end{verbatim}
}

\begin{table}
\centering
\begin{tabular}{l l l}
\toprule
\tabhead{Groups} & \tabhead{Treatment X} & \tabhead{Treatment Y} \\
\midrule
1 & 0.2 & 0.8\\
2 & 0.17 & 0.7\\
3 & 0.24 & 0.75\\
4 & 0.68 & 0.3\\
\bottomrule\\
\end{tabular}
\caption{The effects of treatments X and Y on the four groups studied.}
\label{tab:treatments}
\end{table}

You can reference tables with \verb|\ref{<label>}| where the label is defined within the table environment.
Altenternatively, you can use \verb|\Cref{<label>}|, which will also include the object's type: \Cref{tab:treatments}.

\subsection{Figures}

There will hopefully be many figures in your thesis (that should be placed in the \emph{Figures} folder). The way to insert figures into your thesis is to use a code template like this:
\begin{verbatim}
\begin{figure}
\centering
\includegraphics{figures/electron}
\decoRule
\caption[An Electron]{An electron (artist's impression).}
\label{fig:Electron}
\end{figure}
\end{verbatim}
Also look in the source file. Putting this code into the source file produces the picture of the electron that you can see in the figure below.

\begin{figure}[th]
\centering
\includegraphics{figures/electron}
\decoRule
\caption[An Electron]{An electron (artist's impression).}
\label{fig:Electron}
\end{figure}

Sometimes figures don't always appear where you write them in the source. The placement depends on how much space there is on the page for the figure. Sometimes there is not enough room to fit a figure directly where it should go (in relation to the text) and so \LaTeX{} puts it at the top of the next page. Positioning figures is the job of \LaTeX{} and so you should only worry about making them look good!

Figures usually should have captions just in case you need to refer to them (such as in \Cref{fig:Electron}). The \verb|\caption| command contains two parts, the first part, inside the square brackets is the title that will appear in the \emph{List of Figures}, and so should be short. The second part in the curly brackets should contain the longer and more descriptive caption text.

The \verb|\decoRule| command is optional and simply puts an aesthetic horizontal line below the image. If you do this for one image, do it for all of them.

\LaTeX{} is capable of using images in pdf, jpg and png format.
However, you should generally aim for high-quality vector images (e.g., pdf, or svg), as lower-resolution figures look bad when printed.

Here is another example for an image (cf.~\Cref{fig:scaled-electron}), which has been reduced to fit better on the page:

\begin{figure}[th]
\centering
\includegraphics[width=0.4\columnwidth]{figures/electron}
\decoRule
\caption[A Scaled Electron]{A scaled representation of an electron.}
\label{fig:scaled-electron}
\end{figure}

The code used to render this scaled image is as follows (mind the width parameter passed to \verb|\includegraphics|):
\begin{verbatim}
\begin{figure}[th]
\centering
\includegraphics[width=0.4\columnwidth]{figures/electron}
\decoRule
\caption[A Scaled Electron]{A scaled representation of an electron.}
\label{fig:scaled-electron}
\end{figure}
\end{verbatim}

The next example uses \verb|\minipage| to put two images side by side (cf.~\Cref{fig:side-by-side}):
\begin{figure}[th]
\begin{minipage}[t]{0.5\columnwidth}
    \includegraphics[width=0.45\columnwidth]{figures/electron}
\end{minipage}
\begin{minipage}[t]{0.5\columnwidth}
    \includegraphics[width=0.45\columnwidth]{figures/electron}
\end{minipage}
\caption[A divided, side-by-side image]{A divided, side-by-side image}
\label{fig:side-by-side}
\end{figure}

\subsection{Typesetting mathematics}

If your thesis is going to contain heavy mathematical content, \LaTeX{} can render them without any problems.

The \enquote{Not So Short Introduction to \LaTeX} (available on \href{http://www.ctan.org/tex-archive/info/lshort/english/lshort.pdf}{CTAN}) should tell you everything you need to know for most cases of typesetting mathematics. If you need more information, a much more thorough mathematical guide is available from the AMS called, \enquote{A Short Math Guide to \LaTeX} and can be downloaded from:
\url{ftp://ftp.ams.org/pub/tex/doc/amsmath/short-math-guide.pdf}

There are many different \LaTeX{} symbols to remember, luckily you can find the most common symbols in \href{http://ctan.org/pkg/comprehensive}{The Comprehensive \LaTeX~Symbol List}.

You can write an equation, which is automatically given an equation number by \LaTeX{} like this:
\begin{verbatim}
\begin{equation}
E = mc^{2}
\label{eqn:Einstein}
\end{equation}
\end{verbatim}

This will produce Einstein's famous energy-matter equivalence equation:
\begin{equation}
E = mc^{2}
\label{eqn:Einstein}
\end{equation}

All equations you write (which are not in the middle of paragraph text) are automatically given equation numbers by \LaTeX{}. If you don't want a particular equation numbered, use the unnumbered form:
\begin{verbatim}
\[ a^{2}=4 \]
\end{verbatim}
which will render to
\[ a^{2} = 4 \]

%----------------------------------------------------------------------------------------

\section{Sectioning and Subsectioning}

You should break your thesis up into nice, bite-sized sections and subsections.
\LaTeX{} automatically builds a Table of Contents by looking at all the \verb|\chapter{}|, \verb|\section{}|  and \verb|\subsection{}| commands you write in the source.

The Table of Contents should only list the sections to three (3) levels.
A \verb|chapter{}| is level zero (0). A \verb|\section{}| is level one (1) and so a \verb|\subsection{}| is level two (2). In your thesis it is likely that you will even use a \verb|subsubsection{}|, which is level three (3).
The depth to which the Table of Contents is formatted is set within \file{style.cls}.  
If you need this changed, you can do it in \file{main.tex}.

If you want a chapter, section, or subsection to be skipped from the Table of Contents, you can append an asterisk, like so: \verb|\chapter*{}|, \verb|\section*{}|, and \verb|\subsection*{}|.

%----------------------------------------------------------------------------------------

\section{Source Code Listings}
You can add source code snippets, like on \Cref{lst:c}, \Cref{lst:py}, and \Vref{lst:sh}, using the \verb|\begin{listing}[...]| environment.
The documentation for this package is at \url{https://mirror.physik.tu-berlin.de/pub/CTAN/macros/latex/contrib/listings/listings.pdf}.

\begin{lstlisting}[language=C, caption={C code snippet}, label=lst:c]
int main(int argc, char **argv) {
    int x = 0;
    scanf("%d", &x);
    printf("%d\n", x);
    /* some comment */
    return 0;
}
\end{lstlisting}

\begin{lstlisting}[language=Python, caption={Python code snippet}, label=lst:py]
class A:
    def __init__(x):
        self.x = x

if __name__ == "__main__":
    # another comment
    print("Hello World!")
\end{lstlisting}

\begin{lstlisting}[language=sh, caption={Code snippet in POSIX shell}, label=lst:sh]
#!/bin/sh
echo 'starting'
find . -type f -name '*.bin'
echo 'finished'
exit 0
\end{lstlisting}

%----------------------------------------------------------------------------------------

\section{Acronyms}
% use \ac{...} for singular, and \acp{...} for plural
You can use acronyms like this: The \ac{CPU} is something very important (with \verb|\ac{}|). 
Generally, a computer does not have multiple \acp{CPU} (with \verb|\acp{}| for the plural form).
If you want, you can restate the acronym as if it were mentioned for the first time, like so: \acf{CPU} (with \verb|\acf{}|).

The documentation for this package includes more examples and desribes all usage patterns.
You can find it here: \url{https://mirror.funkfreundelandshut.de/latex/macros/latex/contrib/acronym/acronym.pdf}.

%----------------------------------------------------------------------------------------

\section{Lists and Enumerations}
This is a classic enumeration with one entry per line, done with \verb|\begin{enumerate}|:
\begin{enumerate}
    \item First item;
    \item Second item;
    \item Third item.
\end{enumerate}

This is an inline enumeration with the \verb|\begin{enumerate*}| environment:
\begin{enumerate*}
    \item first item,
    \item second item, and
    \item third item.
\end{enumerate*}

This is a classic list with one entry per line via \verb|\begin{itemize}|:
\begin{itemize}
    \item First item;
    \item Second item;
    \item Third item.
\end{itemize}

This is an inline list of items using \verb|\begin{itemize*}|:
\begin{itemize*}
    \item First item;
    \item Second item;
    \item Third item.
\end{itemize*}

%----------------------------------------------------------------------------------------

\section{Referencing}
With this template you have three ways how to reference labeled entities (i.e., figures, tables, listings, etc.):
\begin{enumerate}
    \item \verb|\ref{}| prints only the entity's id: \ref{fig:Electron}
    \item \verb|\Cref{}| automatically prints the entity's type followed by its id: \Cref{fig:Electron}
    \item \verb|\Vref{}| automatically adds text to describe where the entity is located, so that the reader can find it more easily: \Vref{fig:Electron}.
\end{enumerate}

%----------------------------------------------------------------------------------------

\section{In Closing}
You have reached the end of this mini-guide. 

The easy work of setting up the structure and framework has been taken care of for you. 
It's now your job to fill it out!
Don't forget to also delete/comment out whatever you don't need.

Good luck and have lots of fun!
